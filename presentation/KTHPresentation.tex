\documentclass[t]{beamer}


	% Appearance:
	
	\usetheme{KTHprofile}
	\usefonttheme[onlylarge]{structurebold}
	
	% Standard packages
	
	\usepackage[english]{babel}
	%\usepackage[latin1]{inputenc}
	%\usepackage[utf8]
	\usepackage{times}
	\usepackage[T1]{fontenc}
	\usepackage{lipsum}
	
	% Author, Title, etc.
	
	%\title{}
	
	\title
	{%
	  DT2119 Speech and Speaker Recognition Lab 1
	}
	
	%\author{}
	
	\author[Zifan, Lingxi]
	{
	Zijian Fan, Lingxi Xiong
	}
	
	%\author[Dude, Friends]
	%{
	%The~Dude\inst{1} \and
	%And~Friends\inst{2}
	%}
	
	\institute{}
	
	%\institute[KTH Royal Institute of Technology]
	%{\inst{1} KTH Royal Institute of Technology, Sweden\and
	%\vskip-4mm
	%\inst{2} KTH Royal Institute of Technology, Sweden
	%}
	
	\date{}
	%\date{\today}
	
	
	\begin{document}
	
\begin{frame}
	\titlepage
\end{frame}
	
\begin{frame}
	\frametitle{MFCC -- Enframe}
	\begin{figure}
\centering
		\includegraphics[width=0.48\textwidth]{figures/01.png}
		\includegraphics[width=0.48\textwidth]{figures/02.png}
	\end{figure}
			
	\begin{itemize}
		\item window length = 400(200ms); window shift = 200(100ms)--50\% overlap
		\item increase the resolution and decrease the variance
	\end{itemize}
\end{frame}
	
\begin{frame}
	\frametitle{MFCC -- Pre-emphasis}
	\begin{figure}
		\includegraphics[width=0.6\textwidth]{figures/03.png}
	\end{figure}
			  
	\begin{itemize}
		\item high-pass filter to filter out the low frequency component
		\item $H_{pre}\left( z \right) =1-0.97z^{-1}$
	\end{itemize}
\end{frame}
	
\begin{frame}
	\frametitle{MFCC -- Hamming Window}
	\begin{figure}
\centering
		\includegraphics[width=0.48\textwidth]{figures/03_1.png}
		\includegraphics[width=0.48\textwidth]{figures/04.png}
	\end{figure}

	\begin{itemize}
		\item $h\left[ n \right] =h_I\left[ n \right] \times w\left[ n \right] $
		\item Resolution: $\Delta \upsilon =\text{1.30/}N$
\\
		\item Variance: $Var\{ \widehat{P_{x}^{W}}(v) \} =\frac{9}{8K}P_{x}^2(v) = \frac{9}{8K}Var \{ \widehat{P_x}\left( v \right)\} $
%$Var \left\{ \widehat{P_x}\left( v \right) \right\} = P_{x}^2 (v)$

	\end{itemize}

	
\end{frame}
	
\begin{frame}
	\frametitle{MFCC -- Fast Fourier Transform}
	\begin{figure}
		\includegraphics[width=0.7\textwidth]{figures/05.png}
	\end{figure}
\begin{itemize}
\item $f_{max}$ = 10000Hz
\end{itemize}

\end{frame}
	
\begin{frame}
	\frametitle{MFCC -- Mel filterbank log spectrum}
	\begin{figure}
		\includegraphics[width=0.48\textwidth]{figures/06.png}
		\includegraphics[width=0.48\textwidth]{figures/07.png}
	\end{figure}

	\begin{itemize}
		\item Mel filterbank: filters concentrated in the low frequency area
		%\item discrete signal $\rightarrow$ continuous signal
	\end{itemize}

\end{frame}
	
\begin{frame}
	\frametitle{MFCC -- Cosine Transform and Liftering}
	\begin{figure}
\centering
		\includegraphics[width=0.48\textwidth]{figures/08.png}
		\includegraphics[width=0.48\textwidth]{figures/09.png}
	\end{figure}
		  
	\begin{itemize}
		\item Cosine transform: continous $\rightarrow$ discrete signal
		\item Lifter: correct the range of the coefficients
		\item the MFCCs are similar if from same digits by same speaker, otherwise they vary a lot
	\end{itemize}

\end{frame}
	
\begin{frame}
	\frametitle{Feature Correlation}
	\begin{figure}
		\centering
		\includegraphics[width=0.5\textwidth]{figures/10.png}
		\includegraphics[width=0.49\textwidth]{figures/11.png}
	\end{figure}
		  
	\begin{itemize}
		\item MFCC: uncorrelated features $\rightarrow$ diagonal covariance matrices $\rightarrow$ Gaussian modelling
		\item Mspec: features are much more correlated
	\end{itemize}
\end{frame}
	
\begin{frame}
	\frametitle{Explore Speech Segments with Clustering}
	\begin{figure}
		\centering
		\includegraphics[width=0.5\textwidth]{figures/12.png}
		\includegraphics[width=0.49\textwidth]{figures/13.png}
		\caption{GMM posteriors of utterances containing same words}
	\end{figure}
\begin{itemize}
\item the discovered classes increase with number of components increase

\end{itemize}

\end{frame}

\begin{frame}
	\frametitle{Explore Speech Segments with Clustering}
	\begin{figure}
\centering
		\includegraphics[width=0.5\textwidth]{figures/14.png}
		\includegraphics[width=0.49\textwidth]{figures/15.png}
	\end{figure}
	\begin{itemize}
		\item The classes does not correspond to the phonemes composing each word
		\item Unstable: classes that represent the utterances(word) vary among speakers
	\end{itemize}
\end{frame}


\begin{frame}
	\frametitle{Comparing Utterances -- Global distance}
	\begin{figure}
		\includegraphics[width=0.7\textwidth]{figures/16.png}
		\includegraphics[width=0.29\textwidth]{figures/17.png}
	\end{figure}
	\begin{itemize}
		\item The distance separates digits well even between different speakers
	\end{itemize}

\end{frame}
\begin{frame}
	\frametitle{Comparing Utterances -- hierarchy clustering}
	\begin{figure}
		\includegraphics[width=0.99\textwidth]{figures/18.png}
	\end{figure}
\end{frame}

\begin{frame}
	\frametitle{Thank you for your Attention!}	
\end{frame}
	
\end{document}